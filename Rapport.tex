\documentclass[a4paper]{report}

%====================== PACKAGES ======================

\usepackage[english]{babel}
\usepackage[utf8x]{inputenc}
%pour gérer les positionnement d'images
\usepackage{float}
\usepackage{amsmath}
\usepackage{graphicx}
\usepackage{pdfpages}
\usepackage[colorinlistoftodos]{todonotes}
\usepackage{url}
%pour les informations sur un document compilé en PDF et les liens externes / internes
\usepackage{hyperref}
%pour la mise en page des tableaux
\usepackage{array}
\usepackage{tabularx}
%pour utiliser \floatbarrier
%\usepackage{placeins}
%\usepackage{floatrow}
%espacement entre les lignes
\usepackage{setspace}
%modifier la mise en page de l'abstract
\usepackage{abstract}
%police et mise en page (marges) du document
\usepackage[T1]{fontenc}
\usepackage[top=2cm, bottom=2cm, left=2cm, right=2cm]{geometry}
%Pour les galerie d'images
\usepackage{subfig}

%====================== INFORMATION ET REGLES ======================

%rajouter les numérotation pour les \paragraphe et \subparagraphe
\setcounter{secnumdepth}{4}
\setcounter{tocdepth}{4}

\hypersetup{							% Information sur le document
pdfauthor = {Simon CHANU},			% Auteurs
pdftitle = {Internship Report -
			Obstacle avoidance for an autonomous Sailboat},			% Titre du document
pdfsubject = {},		% Sujet
%pdfkeywords = {Tag1, Tag2, Tag3, ...},	% Mots-clefs
pdfstartview={FitH}}					% ajuste la page à la largueur de l'écran
%pdfcreator = {MikTeX},% Logiciel qui a crée le document
%pdfproducer = {}} % Société avec produit le logiciel

%======================== DEBUT DU DOCUMENT ========================

\begin{document}

%régler l'espacement entre les lignes
\newcommand{\HRule}{\rule{\linewidth}{0.5mm}}

%page de garde
\begin{titlepage}
    \title{ %SHEPHERD REPORT
    		\normalsize \textsc{Project Report}
            \\ [0.5cm]
            \includegraphics[scale=0.5]{image/logo_ensta.png} % Ou autre image
            \\ [1.0cm]
            \HRule{} 
            \\ [1.0cm]
            \LARGE \textbf{\uppercase{SHEPHERD Project}} 
            \\ [0.5cm]
            \HRule{} 
            \\ [0.5cm]
            \includegraphics[scale=0.5]{image/logo_ensta.png}
            %\\
            %\textsc{\textbf{ENSTA Bretagne}}
            \\ [0.5cm]
            \normalsize \today \vspace*{5\baselineskip}
    }


    \date{}

    \author{
            \textbf{Student}\\
            \textsc{ROB} \\
            \textsc{ENSTA Bretagne} \\
            \\
            \textbf{Supervisor}\\
            \textsc{Luc Jaulin}\\
            \textsc{ENSTA Bretagne} 
    }

    \maketitle
\end{titlepage}

%page blanche
\newpage
~
%ne pas numéroter cette page
\thispagestyle{empty}
\newpage

% Abstract
\begin{abstract}

	Abstract

\end{abstract}

\tableofcontents
%ne pas numéroter le sommaire

\newpage

\listoffigures
\thispagestyle{empty}
\setcounter{page}{0}

\newpage

%espacement entre les lignes d'un tableau
\renewcommand{\arraystretch}{1.5}

%====================== INCLUSION DES PARTIES ======================

~
\thispagestyle{empty}
%recommencer la numérotation des pages à "1"
\setcounter{page}{0}
\newpage

\chapter{Chapter}
\section{Section}
\subsection{Subsection}

	\begin{center}
		\includegraphics[scale=0.5]{image/logo_ensta.png}
	\end{center}

	\begin{figure}[!h]
		\centering
		\includegraphics[scale=0.5]{image/logo_ensta.png}\\
		\caption{caption}
	\end{figure} % http://www.boattest.com/view-news/4388_finnish-archipelago-aland-islands-for-exciting-summer-cruising

	%\cite{aland_UAS}
	\footnote{footnote}

\subsection{Subsection}

\section{Section}
\subsection{Subsection}
\subsection{Subsection}

	\begin{itemize}
		\item item1
		\item item2
		\item item2
	\end{itemize}

\chapter{Chapter}
\section{Section}
\subsection{Subsection}
\subsection{Subsection}
\subsection{Subsection}

\section{Section}
\subsection{Subsection}
\subsubsection{Subsubsection}
\subsubsection{Subsubsection}
\subsubsection{Subsubsection}

\paragraph{Paragraph}

	\begin{equation}
		z = A*exp(-t^2)
	\end{equation}
	\begin{equation}
		t = B*((x+a)^2 + (y+b)^2)
	\end{equation}


	\textit{Textit}\\
	Text \\

	\textit{Textit2}\\
	Text2 \\

\vspace{2cm}

	\begin{itemize}
		\item \textbf{Simulate}\\
			Text1
		\item \textbf{Playback}\\
			Text2
		\item \textbf{Log}\\
			Text3
		\item \textbf{Monitor}\\
			Text4
	\end{itemize}

% ===================== Conclusion ====================
\newpage

	\begin{center}
		\textbf{\LARGE{Conclusion}}
	\end{center}

\vspace{5cm}

Conclusion

\newpage
%récupérer les citation avec "/footnotemark"
\nocite{*}

% ====================== Biblio ======================

%choix du style de la biblio
\bibliographystyle{plain}
%inclusion de la biblio
\bibliography{Rapport2A}
%voir wiki pour plus d'information sur la syntaxe des entrées d'une bibliographie


% ====================== Annexes ======================
\newpage

	\begin{center}
		\title{ \HRule{} \\ [12cm]
				\LARGE \textbf{\uppercase{Annexes}}\\ [12cm]
				\HRule{}}
		\maketitle
	\end{center}

\end{document}
